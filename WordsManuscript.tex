\documentclass[man]{apa6}
\usepackage{lmodern}
\usepackage{amssymb,amsmath}
\usepackage{ifxetex,ifluatex}
\usepackage{fixltx2e} % provides \textsubscript
\ifnum 0\ifxetex 1\fi\ifluatex 1\fi=0 % if pdftex
  \usepackage[T1]{fontenc}
  \usepackage[utf8]{inputenc}
\else % if luatex or xelatex
  \ifxetex
    \usepackage{mathspec}
  \else
    \usepackage{fontspec}
  \fi
  \defaultfontfeatures{Ligatures=TeX,Scale=MatchLowercase}
\fi
% use upquote if available, for straight quotes in verbatim environments
\IfFileExists{upquote.sty}{\usepackage{upquote}}{}
% use microtype if available
\IfFileExists{microtype.sty}{%
\usepackage{microtype}
\UseMicrotypeSet[protrusion]{basicmath} % disable protrusion for tt fonts
}{}
\usepackage{hyperref}
\hypersetup{unicode=true,
            pdftitle={Ambiguous Words},
            pdfauthor={Nicholas R. Harp, Catherine C. Brown, \& Maital Neta},
            pdfkeywords={ambiguity},
            pdfborder={0 0 0},
            breaklinks=true}
\urlstyle{same}  % don't use monospace font for urls
\usepackage{graphicx,grffile}
\makeatletter
\def\maxwidth{\ifdim\Gin@nat@width>\linewidth\linewidth\else\Gin@nat@width\fi}
\def\maxheight{\ifdim\Gin@nat@height>\textheight\textheight\else\Gin@nat@height\fi}
\makeatother
% Scale images if necessary, so that they will not overflow the page
% margins by default, and it is still possible to overwrite the defaults
% using explicit options in \includegraphics[width, height, ...]{}
\setkeys{Gin}{width=\maxwidth,height=\maxheight,keepaspectratio}
\IfFileExists{parskip.sty}{%
\usepackage{parskip}
}{% else
\setlength{\parindent}{0pt}
\setlength{\parskip}{6pt plus 2pt minus 1pt}
}
\setlength{\emergencystretch}{3em}  % prevent overfull lines
\providecommand{\tightlist}{%
  \setlength{\itemsep}{0pt}\setlength{\parskip}{0pt}}
\setcounter{secnumdepth}{0}
% Redefines (sub)paragraphs to behave more like sections
\ifx\paragraph\undefined\else
\let\oldparagraph\paragraph
\renewcommand{\paragraph}[1]{\oldparagraph{#1}\mbox{}}
\fi
\ifx\subparagraph\undefined\else
\let\oldsubparagraph\subparagraph
\renewcommand{\subparagraph}[1]{\oldsubparagraph{#1}\mbox{}}
\fi

%%% Use protect on footnotes to avoid problems with footnotes in titles
\let\rmarkdownfootnote\footnote%
\def\footnote{\protect\rmarkdownfootnote}


  \title{Ambiguous Words}
    \author{Nicholas R. Harp\textsuperscript{1}, Catherine C. Brown\textsuperscript{1}, \& Maital Neta\textsuperscript{1}}
    \date{}
  
\shorttitle{AMBIGUOUS WORDS}
\affiliation{
\vspace{0.5cm}
\textsuperscript{1} University of Nebraska-Lincoln}
\keywords{ambiguity\newline\indent Word count: X}
\usepackage{csquotes}
\usepackage{upgreek}
\captionsetup{font=singlespacing,justification=justified}

\usepackage{longtable}
\usepackage{lscape}
\usepackage{multirow}
\usepackage{tabularx}
\usepackage[flushleft]{threeparttable}
\usepackage{threeparttablex}

\newenvironment{lltable}{\begin{landscape}\begin{center}\begin{ThreePartTable}}{\end{ThreePartTable}\end{center}\end{landscape}}

\makeatletter
\newcommand\LastLTentrywidth{1em}
\newlength\longtablewidth
\setlength{\longtablewidth}{1in}
\newcommand{\getlongtablewidth}{\begingroup \ifcsname LT@\roman{LT@tables}\endcsname \global\longtablewidth=0pt \renewcommand{\LT@entry}[2]{\global\advance\longtablewidth by ##2\relax\gdef\LastLTentrywidth{##2}}\@nameuse{LT@\roman{LT@tables}} \fi \endgroup}


\DeclareDelayedFloatFlavor{ThreePartTable}{table}
\DeclareDelayedFloatFlavor{lltable}{table}
\DeclareDelayedFloatFlavor*{longtable}{table}
\makeatletter
\renewcommand{\efloat@iwrite}[1]{\immediate\expandafter\protected@write\csname efloat@post#1\endcsname{}}
\makeatother
\usepackage{lineno}

\linenumbers

\authornote{Nicholas R. Harp, Department of Psychology, Center for Brain, Biology, and Behavior, University of Nebraska-Lincoln
Catherine C. Brown, Department of Psychology, Center for Brain, Biology, and Behavior, University of Nebraska-Lincoln
Maital Neta, Department of Psychology, Center for Brain, Biology, and Behavior, University of Nebraska-Lincoln

Correspondence concerning this article should be addressed to Nicholas R. Harp, Postal address. E-mail: \href{mailto:nharp@huskers.unl.edu}{\nolinkurl{nharp@huskers.unl.edu}}}

\abstract{
We found some ambiugous words.

Two or three sentences explaining what the \textbf{main result} reveals in direct comparison to what was thought to be the case previously, or how the main result adds to previous knowledge.

One or two sentences to put the results into a more \textbf{general context}.

Two or three sentences to provide a \textbf{broader perspective}, readily comprehensible to a scientist in any discipline.


}

\begin{document}
\maketitle

\hypertarget{introduction}{%
\section{Introduction}\label{introduction}}

We wanted to identify ambiguous words. Mention in this section why we also generated clearly positive and negative words.

\hypertarget{study-1-pilot}{%
\section{Study 1: Pilot}\label{study-1-pilot}}

\hypertarget{methods}{%
\subsection{Methods}\label{methods}}

\hypertarget{participants}{%
\subsubsection{Participants}\label{participants}}

Workers on Amazon's Mchanical Turk (MTurk) were invted to participate in an eligibility screener worth \$0.20 with the option to earn a bonus of \$2.05 if they met the requirements and completed the entire study. See \textbf{Supplementary Information} for specific MTurk batch settings. The Workers clicked a hyperlink that directed them to the study. The screener task included demographic questions and one block of word ratings that included 5 instances of the word \enquote{negative} and 5 instances of the word \enquote{positive} (see Procedure below for full details). Workers were invited to complete the entire study if they indicated that they were over 18 years old, had English as their native language, had no history of psychological or neurological disorder, and correctly rated the words \enquote{positive} and \enquote{negative} as positive or negative with at least 80\% accuracy. Of the 145 Workers who completed the screener, 119 met the eligibility requirements, and 103 (54.37\% female, 45.63\% male) chose to complete the entire study. The final sample was 3.88\% Asian, 5.83\% Black, 85.44\% White, with a mean(sd) age of 37.16(10.60).

\hypertarget{material}{%
\subsubsection{Material}\label{material}}

\hypertarget{stimuli}{%
\paragraph{Stimuli}\label{stimuli}}

We compiled an initial set of 59 words that we believed had two distinct definitions, one clearly positive definition and one clearly negative definition. To create lists of clearly positive and clearly negative words, we first created a master list of words that were included in both the study by Warriner, Kuperman, and Brysbaert (2013), for valence and arousal ratings, and the Enlgish Lexicon Project online word query (Balota et al., 2007), for lexical characterisic measurements. We then elimiated any words with a mean arousal rating that was greater than 1 standard deviation away from the mean arousal of the list of 59 ambiguous words. We classified \enquote{positive} words as those with a mean valence \textgreater{} 7 on the 1-9 scale used by Warriner et al. (2013); \enquote{negative} words had mean valence \textless{} 3. To ensure that all words shared similar lexical characteristics, we eliminated any words from the master list whose lexical characteristics did not fall within the minimum and maximum values of the 59 ambiguous words' lexical characteristics. The following were used for the cutoffs: length, the frequency of a word as reported by the Hyperspace Analogue to Language (HAL) study (Lund \& Burgess, 1996), the log of HAL frequency, number of phonemes, number of syllables, number of morphemes, lexical decision reaction time and accuracy, and naming reaction time and accuracy. The final list of pilot words included 59 ambiguous, 267 positive, and 304 negative words.

All of the calculations described in this section were scripted using R version 3.6.1 and are available in the \textbf{Supplementary Information}.

\hypertarget{software}{%
\paragraph{Software}\label{software}}

All tasks were created and presented using Gorilla Experiment Builder (Anwyl-Irvine, Massonnié, Flitton, Kirkham, \& Evershed, 2019). The study was only accessible to participants using a computer (not a phone or tablet) within the United States.

\hypertarget{procedure}{%
\subsubsection{Procedure}\label{procedure}}

\hypertarget{screener-and-word-rating-task}{%
\paragraph{Screener and word rating task}\label{screener-and-word-rating-task}}

After giving informed consent, participants first answered demographic questions about their gender, age, race, native language, and whether they had ever been diagnosed with a psychological or neurological disorder. They then were shown a brief self-guided instructional walkthrough of the task before completing the screener.

Using a random seed, we selected 20 positive and 20 negative words from the final pilot list for use in the screener task. These 40 words, along with 5 instances of the word \enquote{positive} and 5 instances of the word \enquote{negative} were presented randomly, one at a time, each following a 250 ms fixation cross. Each word remained on screen until the participant indicated that they thought it was positive or negative by pressing A or L on their keyboard (key pairing randomized across participants). If no response was made after 2000ms, a reminder appeared on screen, \enquote{Please respond as quickly as you can! A = POSITIVE. L = NEGATIVE.} Participants who rated the words \enquote{positive} and \enquote{negative} with less than 80\% accuracy were compensated for their time but were not invited to complete the rest of the study. Participants were also excluded at this point if they indicated that they were younger than 18, that English was not their native language, or that they had been diagnosed with a psychological or neurological disorder.

The remaining 590 words from the final pilot list were randomly presented across 10 blocks of 59 words using the same button-press procedure as the screener block.

\hypertarget{data-preprocessing}{%
\subsubsection{Data preprocessing}\label{data-preprocessing}}

Trials with a response time faster than 250ms (n = ) were removed from the data prior to analysis, as well as trials greater than 3 SDs above the mean reaction time averaged across all trials.

\hypertarget{results}{%
\subsection{Results}\label{results}}

\hypertarget{subjective-ratings}{%
\subsubsection{Subjective ratings}\label{subjective-ratings}}

\hypertarget{reaction-times}{%
\subsubsection{Reaction times}\label{reaction-times}}

\hypertarget{study-2-comparison-of-words-with-valence-bias-and-ipanat}{%
\section{Study 2: Comparison of words with valence bias and IPANAT}\label{study-2-comparison-of-words-with-valence-bias-and-ipanat}}

\hypertarget{methods-1}{%
\subsection{Methods}\label{methods-1}}

\hypertarget{participants-1}{%
\subsubsection{Participants}\label{participants-1}}

\hypertarget{material-1}{%
\subsubsection{Material}\label{material-1}}

\hypertarget{stimuli-1}{%
\paragraph{Stimuli}\label{stimuli-1}}

\hypertarget{valence-bias-with-words}{%
\subparagraph{Valence Bias with Words}\label{valence-bias-with-words}}

\hypertarget{valence-bias-with-faces}{%
\subparagraph{Valence Bias with Faces}\label{valence-bias-with-faces}}

\hypertarget{valence-bias-with-iaps}{%
\subparagraph{Valence Bias with IAPS}\label{valence-bias-with-iaps}}

\hypertarget{ipanat}{%
\subparagraph{IPANAT}\label{ipanat}}

\hypertarget{software-1}{%
\paragraph{Software}\label{software-1}}

\hypertarget{procedure-1}{%
\subsubsection{Procedure}\label{procedure-1}}

\hypertarget{valence-bias-tasks}{%
\paragraph{Valence Bias Tasks}\label{valence-bias-tasks}}

\hypertarget{ipanat-1}{%
\paragraph{IPANAT}\label{ipanat-1}}

\hypertarget{data-analysis}{%
\subsubsection{Data analysis}\label{data-analysis}}

\hypertarget{valence-bias-tasks-1}{%
\paragraph{Valence Bias Tasks}\label{valence-bias-tasks-1}}

\hypertarget{ipanat-2}{%
\paragraph{IPANAT}\label{ipanat-2}}

\hypertarget{results-1}{%
\subsection{Results}\label{results-1}}

\hypertarget{subjective-ratings-1}{%
\subsubsection{Subjective ratings}\label{subjective-ratings-1}}

\hypertarget{valence-bias-with-words-1}{%
\subparagraph{Valence Bias with Words}\label{valence-bias-with-words-1}}

\hypertarget{valence-bias-with-faces-1}{%
\subparagraph{Valence Bias with Faces}\label{valence-bias-with-faces-1}}

\hypertarget{valence-bias-with-iaps-1}{%
\subparagraph{Valence Bias with IAPS}\label{valence-bias-with-iaps-1}}

\hypertarget{ipanat-3}{%
\subparagraph{IPANAT}\label{ipanat-3}}

\hypertarget{reaction-times-1}{%
\subsubsection{Reaction times}\label{reaction-times-1}}

\hypertarget{valence-bias-with-words-2}{%
\subparagraph{Valence Bias with Words}\label{valence-bias-with-words-2}}

\hypertarget{valence-bias-with-faces-2}{%
\subparagraph{Valence Bias with Faces}\label{valence-bias-with-faces-2}}

\hypertarget{valence-bias-with-iaps-2}{%
\subparagraph{Valence Bias with IAPS}\label{valence-bias-with-iaps-2}}

\hypertarget{ipanat-4}{%
\subparagraph{IPANAT}\label{ipanat-4}}

\hypertarget{relationships-between-the-measures}{%
\subsubsection{Relationships between the measures}\label{relationships-between-the-measures}}

\hypertarget{discussion}{%
\section{Discussion}\label{discussion}}

We did this study.

\newpage

\hypertarget{references}{%
\section{References}\label{references}}

\begingroup
\setlength{\parindent}{-0.5in}
\setlength{\leftskip}{0.5in}

\hypertarget{refs}{}
\leavevmode\hypertarget{ref-anwyl-irvine_gorilla_2019}{}%
Anwyl-Irvine, A. L., Massonnié, J., Flitton, A., Kirkham, N., \& Evershed, J. K. (2019). Gorilla in our midst: An online behavioral experiment builder. \emph{Behav Res}. \url{https://doi.org/10.3758/s13428-019-01237-x}

\leavevmode\hypertarget{ref-balota_english_2007}{}%
Balota, D. A., Yap, M. J., Hutchison, K. A., Cortese, M. J., Kessler, B., Loftis, B., \ldots{} Treiman, R. (2007). The English Lexicon Project. \emph{Behavior Research Methods}, \emph{39}(3), 445--459. \url{https://doi.org/10.3758/BF03193014}

\leavevmode\hypertarget{ref-lund_producing_1996}{}%
Lund, K., \& Burgess, C. (1996). Producing high-dimensional semantic spaces from lexical co-occurrence. \emph{Behavior Research Methods, Instruments, \& Computers}, \emph{28}(2), 203--208. \url{https://doi.org/10.3758/BF03204766}

\leavevmode\hypertarget{ref-warriner_norms_2013}{}%
Warriner, A. B., Kuperman, V., \& Brysbaert, M. (2013). Norms of valence, arousal, and dominance for 13,915 English lemmas. \emph{Behav Res}, \emph{45}(4), 1191--1207. \url{https://doi.org/10.3758/s13428-012-0314-x}

\endgroup

\newpage

\hypertarget{supplementary-information}{%
\section{Supplementary Information}\label{supplementary-information}}

\hypertarget{mturk-project-settings}{%
\subsection{MTurk Project Settings}\label{mturk-project-settings}}

The following are the settings used for the first batch on MTurk. This batch only contributed 6 respondents because the batch was published before the Gorilla task was fully functioning, and the batch expired before all HITs could be filled.

\begin{itemize}
\tightlist
\item
  Title: Screener: Rate words as positive or negative (WARNING: This HIT may contain adult content. Worker discretion is advised.)\\
\item
  Description: Bonus available (\$2.05) to those who meet eligibility. Complete short demographic questions. Use your keyboard to indicate if you think individual words are positive or negative.
\item
  Keywords: survey, demographics, rating, rate, words
\item
  Reward per response: \$0.2
\item
  Number of respondents: 9
\item
  Time allotted per worker: 1 Hour
\item
  Survey expires in: 7 Days
\item
  Auto-approve and pay Workers in: 3 Days
\item
  Require that Workers be Masters to do your tasks: Yes
\item
  Specify any additional qualifications Workers must meet to work on your tasks:

  \begin{itemize}
  \tightlist
  \item
    Location is UNITED STATES (US)
  \item
    HIT Approval Rate (\%) for all Requesters' HITs greater than 95
  \item
    Number of HITs Approved greater than 5000
  \end{itemize}
\item
  Project contains adult content: selected
\item
  Task Visibility: Hidden - Only Workers that meet my Qualification requirements can see and preview my tasks
\end{itemize}

The same settings were used for the rest of the batches except that they did not require that Workers be Masters and the Number of HITs Approved was set to greater than 500, not 5000.

\hypertarget{study-1-stimuli}{%
\subsection{Study 1 Stimuli}\label{study-1-stimuli}}

Insert link to repository for \enquote{pick\_words.R}


\end{document}
